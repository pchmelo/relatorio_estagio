\section{Metodologia utilizada e principais atividades desenvolvidas}

Esta secção descreve a metodologia adotada durante o estágio, o que inclui
o desenvolvimento iterativo, ferramentas de comunicação e colaboração, \textit{stakeholders}
e suas responsabilidades e as atividades que foram realizadas durante o estágio.

\subsection{Metodologia utilizada}
O estágio foi conduzido com base em uma abordagem de desenvolvimento iterativo, caracterizada pela definição de 
objetivos diários e pela execução das atividades no ambiente do laboratório. Ao final de cada jornada, era realizado 
o recebimento de \textit{feedback} por parte do orientador, o que possibilitou ajustes contínuos e aprimoramentos no 
trabalho desenvolvido.

Adicionalmente, algumas tarefas foram realizadas fora do horário regular de estágio, com o intuito de dar continuidade
ao progresso do projeto. Essas atividades incluíram ajustes pontuais em tarefas previamente desenvolvidas, elaboração
de documentação e realização de pesquisas complementares relacionadas ao tema. \\*

Ao longo do dia, eram realizadas, em média, três reuniões principais. A primeira acontecia pela manhã, com o objetivo de 
realizar um ponto de situação do projeto, fornecer \textit{feedback} sobre o trabalho eventualmente desenvolvido durante a semana e 
definir os principais objetivos a serem alcançados no decorrer do dia. A segunda reunião ocorria no início da tarde, 
momento em que se discutia o progresso das atividades realizadas durante a manhã. A terceira reunião era realizada ao 
final do dia, com o propósito de revisar as tarefas executadas e planear as próximas etapas do projeto para a semana 
seguinte. Nessa reunião, também eram definidos os trabalhos complementares a serem realizados fora do horário de estágio.

O orientador esteve sempre disponível para oferecer suporte e esclarecimentos, tanto durante os dias presenciais no 
laboratório quanto fora do horário regular, por meio de diferentes canais de comunicação.

Em casos mais técnicos ou que exigiam uma análise mais aprofundada, também havia sempre a possibilidade de consultar outros engenheiros do laboratório, que tiveram sempre disponibilidade 
para ajudar.

Toda a comunicação no âmbito do laboratório foi realizada presencialmente. Nos casos em que se fez necessária a comunicação 
à distância, utilizou-se o e-mail institucional como meio oficial.

Como o projeto era muito abrangente, a comunicação foi fundamental para o alinhamento das expectativas e progresso do trabalho. Ao longo do estágio,
todos os requisitos que foram estabelecidos foram ajustados sempre às necessidades da instituição como também às expectativas do estagiário.
Desta forma, todo o projeto foi desenvolvido de maneira colaborativa e adaptativa, garantindo que as metas fossem alcançadas de forma eficaz.

Todo o código desenvolvido durante o estágio foi inicialmente armazenado em um repositório pessoal no GitHub, sendo 
posteriormente compartilhado com a equipe por meio de um repositório no GitLab.
	
	
\subsection{Intervenientes, papéis e responsabilidades}
Este estágio envolveu a colaboração de diferentes participantes, cada um com responsabilidades específicas:

\begin{itemize}
    \item Orientador da instituição:
    \begin{itemize}
        \item[-] Engenheiro Pedro Pascoal: Responsável por orientar o estagiário por meio de supervisão contínua e dar \textit{feedback} construtivo em 
        relação ao trabalho desenvolvido. Também auxiliou na integração do estagiário às instalações da instituição.
    \end{itemize}
    \item Tutor da FEUP/FCUP:
    \begin{itemize}
        \item[-] Professor Rolando Martins: Responsável por orientar o estagiário do lado mais académico representando os objetivos da unidade curricular e na elaboração
        do relatório final.
    \end{itemize}
    \item Cliente:
    \begin{itemize}
        \item[-] Engenheiro Pedro Pascoal: Responsável por representar os interesses do INESC TEC, estabelendo os requisitos e expectativas do projeto. Também foi
        responsável por validar o trabalho desenvolvido.
    \end{itemize}
    \item Estagiário:
    \begin{itemize}
        \item[-] Vasco Melo: Responsável por desenvolver as diferentes tarefas propostas pelo cliente e tendo em conta a qualidade e o tempo proposto para a conclusão das mesmas.
        Também é responsável por relatar todo o estágio através dos diferentes elementos de avaliação propostos pela unidade curricular.
    \end{itemize}
\end{itemize}

\subsection{Atividades desenvolvidas}

Durante o estágio, foram desenvolvidas diferentes atividades, que incluiram  planeamento do projeto, pesquisa de mercado, e desenvolvimento de protótipos.
Ao longo que novos requisitos foram surgindo, as atividades foram ajustadas para atender às novas demandas do projeto. Desta forma, as atividades desenvolvidas
podem ser descritas da seguinte forma:
\begin{itemize}
    \item[] 1. Planeamento do projeto e pesquisa de mercado:
        \begin{itemize}
            \item[.] Apresentação ao estagiário das instalações e seus equipamentos;
            \item[.] Estabelecimento de um plano de ação para o desenvolvimento do projeto;
            \item[.] Foi realizada uma pesquisa de mercado com base nos equipamentos disponíveis e os objetivos esperados, que foram sendo estabelecidos/ajustados 
            ao longo do projeto.
        \end{itemize}
    \item[] 2. Raspberry Pi 3 e Sensores DS18B20:
        \begin{itemize}
            \item[.] Configuração do Raspberry Pi 3;
            \item[.] Desenvolvimento de um sistema de leitura de dados com os sensores DS18B20.
        \end{itemize}
    \item[] 3. Criação de uma GUI e sistema de comunicação entre a RPI e o MainPC: 
        \begin{itemize}
            \item[.] Levantamento de requisitos para a interface gráfica e de tecnologias no mercado;
            \item[.] Desenvolvimento de diferentes protótipos para a GUI;
            \item[.] Estabelecimento de um protocolo SSL/TLS com a RPI;
            \item[.] Criação de um protocolo de controlo de teste de temperatura com a Raspberry Pi 3 e o MainPC.
        \end{itemize}
    \item[] 4. Montar um sistema de gestão para armazenamento dos testes:
        \begin{itemize}
            \item[.] Estudo sobre os dados a guardar;
            \item[.] Criação do schema para a base de dados;
            \item[.] Implementação de um sistema de controlo da base de dados.
        \end{itemize}
    \item[] 5. Implementação de um sistema de migração de base de dados:
        \begin{itemize}
            \item [.] Criação de um sistema de migração de base de dados entre o MainPC e a RPI;
            \item [.] Criação de um sistema de migração de base de dados entre a RPI e o EVSE.
        \end{itemize}
    \item[] 5. Comunicação com o EVSE:
        \begin{itemize}
            \item[.] Estabelecimento da comunicação via SSH entre a RPI e o EVSE.
        \end{itemize}
    \item[] 7. Criar uma dashboard para visualização dos dados:
        \begin{itemize}
            \item[.] Estudo sobre requisitos e funcionalidades da dashboard;
            \item[.] Análise das tecnologias disponíveis para a implementação;
            \item[.] Teste das diferentes tecnologias;
            \item[.] Implementação da dashboard com as funcionalidades definidas.
        \end{itemize}
    \item[] 8. Sistema de schedule de migração de testes: 
        \begin{itemize}
            \item[.] Estabelecimento de um sistema de schedule entre o MainPC e a RPI;
            \item[.] Estabelecimento de um sistema de schedule entre a RPI e o EVSE.
        \end{itemize}
    \item[] 9. Sistema de estabilização e comunicação com a câmara termal:
        \begin{itemize}
            \item[.] Estudo sobre as funcionalidades da câmara termal;
            \item[.] Estabelecimento da comunicação entre a RPI e a câmara termal via serial port;
            \item[.] Investigação sobre diferentes formulações matemáticas de deteção de estabilização de temperatura da câmara termal;
            \item[.] Implementação de um sistema de deteção de estabilização de temperatura com a câmara termal.
        \end{itemize}
    \item[] 10. Documentação :
        \begin{itemize}
            \item[.] Elaboração de diferentes diagramas UML para o sistema desenvolvido;
            \item[.] Criação de documentação técnica para o sistema desenvolvido.
        \end{itemize}
    \item[] 11. Bootstrapping:
        \begin{itemize}
            \item[.] Elaboração de um ficheiro dockerfile para facilitar na instalação da componente MainPC;
            \item[.] Elaboração de um ficheiro .bash e um sistema ansible para facilitar a instalação da componente RPI.
        \end{itemize}
\end{itemize}

Para melhor organização do trabalho desenvolvido, foi elaborado um gantt chart, que pode ser visto na Figura -----------.

TODO: Colocar o gantt chart aqui