\section{Introdução}
Este relatório tem como objetivo apresentar todo o trabalho desenvolvido 
durante o estágio curricular, onde foi desenvolvido um protocolo de teste de 
desempenho de carregadores de energia.

\subsection{Enquadramento}
O estágio curricular foi realizado no Laboratório de Redes Elétricas Inteligentes e Veículos Elétricos (SGEV) do 
Centro de Sistemas de Energia (CPES) do Instituto de Engenharia de Sistemas e Computadores, Tecnologia e Ciência 
(INESC TEC). Para realização do estágio teve como orientador da instituição o engenheiro Pedro Pascoal e como 
orientador da FEUP/FCUP o professor Rolando Martins. \\*

\noindent O SGEV é um espaço dedicado à pesquisa e desenvolvimento de tecnologias inovadoras na área dos sistemas de 
energia, com principal foco na eficiência energética, mobilidade elétrica, redes elétricas inteligentes ou 
energias renováveis.
Com a crescente demanda por soluções de carregamento de veículos elétricos
(VEs), a eficiência e a segurança dos carregadores são aspectos cruciais. O INESC TEC
atualmente desenvolve carregadores CA que operam em ambientes externos sob
exposição de intempéries, podendo ocasionar uma elevação significativa na sua
temperatura interna, especialmente no verão. Diante desse cenário, é fundamental realizar
testes para avaliar o impacto das variações de temperatura externa sobre o desempenho
dos carregadores. Esses testes permitem compreender como a temperatura afeta o
funcionamento e a eficiência dos carregadores, assegurando a confiabilidade dos
produtos em condições reais de operação.

	
\subsection{Objetivos e resultados esperados}
Desenvolver e implementar um sistema de medição de temperatura com sensores DS18B20 e Raspberry Pi 3, que será 
utilizado para monitorar o desempenho térmico de carregadores de veículos elétricos em diferentes condições de 
temperatura, simuladas em uma câmara climática.

Ao final do estágio, espera-se que o protocolo de medição esteja implementado e validado, permitindo a coleta de 
dados precisos sobre o desempenho térmico dos carregadores, os quais serão apresentados em uma dashboard.
\subsection{Estrutura do relatório}
O relatório está estruturado da seguinte forma:
\begin{itemize}
    \item \textbf{Capítulo 1: Introdução} - Apresenta o contexto do estágio, os objetivos e a estrutura do relatório;
    \item \textbf{Capítulo 2: Metodologia} - Descreve a metodologia utilizada durante o estágio curricular,
    os intervenientes (seu papel e responsabilidade) e atividades desenvolvidas;
    \item \textbf{Capítulo 3: Desenvolvimento} - Detalha o desenvolvimento do protocolo , requisitos (funcionais e não funcionais), 
    arquitetura e tecnologias do sistema, solução desenvolvida (dividida nos diferentes módulos) e validação;
    \item \textbf{Capítulo 4: Conclusão} - Apresenta os resultados alcançados, lições aprendidas e trabalho futuro.
\end{itemize}