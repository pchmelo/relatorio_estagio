\section{Introduction}

This report aims to present all the work carried out during the curricular 
internship, which was conducted within the scope of the Capstone Project
course of the Bachelor’s Degree in Informatics and Computing Engineering at 
the Faculty of Engineering of the University of Porto (FEUP) and the Faculty of Sciences of the University of Porto (FCUP).


\subsection{Context}
The curricular internship was conducted at the Smart Grids and Electric Vehicles Laboratory (SGEV) of the Power Systems Center (CPES) at the 
Institute for Systems and Computer Engineering, Technology and Science (INESC TEC). The internship was supervised by engineer Pedro Pascoal from 
the institution and by professor Rolando Martins from FEUP/FCUP.

INESC TEC is a non-profit Portuguese research and development institute 
involved in European projects, specializing in areas such as robotics, 
artificial intelligence, energy, and telecommunications.

SGEV is a space dedicated to research and development of innovative technologies in the field of energy systems, with a primary focus on energy 
efficiency, electric mobility, smart grids, and renewable energy.
With the growing demand for electric vehicle (EV) charging solutions, the efficiency and safety of chargers are crucial aspects. INESC TEC currently develops 
AC chargers that operate in outdoor environments under weather exposure, which can cause a significant increase in their internal temperature, 
especially during summer. Given this scenario, it is essential to conduct tests to evaluate the impact of external temperature variations on 
charger performance. These tests enable understanding of how temperature affects the operation and efficiency of chargers, ensuring 
product reliability under real operating conditions.

\subsection{Objectives and expected results}
To develop and implement a temperature measurement system using DS18B20 sensors and Raspberry Pi 3, which will be used to monitor the thermal 
performance of electric vehicle chargers under different temperature conditions, simulated in a climatic chamber.

At the end of the internship, it is expected that the measurement protocol will be implemented and validated, enabling the collection of precise data on 
the thermal performance of chargers, which will be presented in a dashboard.

\subsection{Report structure}
The report is structured as follows:
\begin{itemize}
    \item \textbf{Chapter 1: Introduction} - Presents the internship context, objectives, and report structure;
    \item \textbf{Chapter 2: Methodology} - Describes the methodology used during the curricular internship, the stakeholders (their role and responsibility), and activities developed;
    \item \textbf{Chapter 3: Development} - Details the protocol development, requirements (functional and non-functional), system architecture and technologies, developed solution (divided into different modules), and validation;
    \item \textbf{Chapter 4: Conclusion} - Presents the achieved results, lessons learned, and future work.
\end{itemize}