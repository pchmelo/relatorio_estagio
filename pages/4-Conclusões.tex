\section{Conclusions}


\subsection{Achieved Results} 	
The temperature testing protocol developed achieved the proposed objectives, enabling the monitoring and control of 
electric vehicle charging, with real-time and easy monitoring of the obtained results. Furthermore, it was possible to verify the 
efficiency of electric car charging over time, allowing performance analysis of the EVSE charger under different temperature 
conditions.
Additionally, the protocol was developed in a modular fashion, which allows it to be easily adapted to different types of devices,
sensors, databases, etc. For a research laboratory context, the protocol is easily adaptable to different 
research contexts. 

This protocol will be an important aid in the development of new electric vehicle chargers, as a means of validation.


\subsection{Lessons Learned} 	
The internship provided a deeper insight into research work in a laboratory context, where several projects
are being developed simultaneously. Furthermore, since the proposed problem was applicable to a real context and was 
developed from scratch, it was possible to learn several lessons about software development, such as the importance of having good
documentation, the importance of having good communication with the team, and the importance of having good work organization.
I am very grateful for the opportunity to have worked with such an experienced and dedicated team, which helped me 
grow as a professional. 
	
\subsection{Future Work} 
Future work would involve improving the protocol to provide more relevant information about charging
that would help in the performance analysis of the EVSE charger, such as being able to perform multiple charging 
sessions during the same test. 
Another interesting feature would be the possibility of comparing the obtained results with results from other conducted tests.
Furthermore, it would be interesting to implement a REST API to allow the protocol to be easily integrated with other systems and applications.
	
\subsection{Acknowledgments}
In this section, I would like to thank everyone who contributed to making the internship a success. First, I would like to 
thank INESC TEC for the opportunity to carry out the internship at the SGEV laboratory, where I had the opportunity to learn and grow as 
a professional. I would also like to thank my supervisor at the institution, engineer Pedro Pascoal, for his support and guidance 
throughout the internship and for providing me with the best means and resources for the development of the work, in addition to all his patience 
and understanding throughout the process. I would like to thank my supervisor at FEUP/FCUP, professor Rolando Martins, for his support and 
guidance throughout the internship in the academic aspect and for helping me with the approval and suggestion of improvements to the project. I would like to 
thank all my internship colleagues, who helped and supported me throughout the internship, and who made the work environment more pleasant.
Finally, I would like to thank the Faculty of Engineering of the University of Porto (FEUP) and the Faculty of Sciences of the University of 
Porto (FCUP) for the opportunity to develop this project and to learn from the best during my degree.