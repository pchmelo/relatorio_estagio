
\section{Conclusões}


\subsection{Resultados alcançados} 	
O protocolo de teste de temperatura desenvolvido atinjiu os objetivos propostos, permitindo a monitorização e controlo do carregamento de 
veículos elétricos, monitorização em tempo real e fácil dos resultados obtidos em tempo real. Além disso, foi possível verificar a eficiência 
do carregamento do carro elétrico ao longo do tempo, permitindo a análise de desempenho do carregador EVSE em diferentes condições de 
temperatura.
Além disso, o protocolo foi desenvolvido de uma forma modular o que permite que seja facilmente adaptado a diferentes tipos de dispositivos,
sensores, base de dados, etc. Para um contexto de laboratório de investigação, o protocolo é facilmente adaptável a diferentes 
contextos de investigação. 

Este protocolo irá ser uma ajuda importante no desenvolvimento de novos carregadores de veículos elétricos, como forma de validação do mesmo.


\subsection{Lições aprendidas} 	
O estágio permitiu ter uma visão mais aprofundada sobre o trabalho de investigação em um contexto de laboratório, onde vários projetos
estão a ser desenvolvidos em simultâneo. Além disso, como o problema proposto era um problema aplicável a um contecto real e ser 
desenvolvido do zero, foi possível aprender várias lições sobre o desenvolvimento de software, como a importância de ter uma boa
documentação, a importância de ter uma boa comunicação com a equipa e a importância de ter uma boa organização do trabalho.
Estou muito grato pela oportunidade de ter trabalhado com uma equipa tão experiente e dedicada, que me ajudou a 
crescer como profissional. 
	
\subsection{Trabalho futuro} 
O trabalho futuro passaria por melhorar o protocolo de forma a fornecer mais informações relevantes sobre o carregamento
que ajudariam na ánálise de desempenho do carregador EVSE, como por exemplo, poder realizar diversas sessões de 
carregamento durante o mesmo teste. 
Outra featura interessante seria a possibilidade de comparar os resultados obtidos com os resultados de outros testes realizados.
Além disso, seria interessante implementar uma REST API para permitir que o protocolo seja facilmente integrado com outros sistemas e aplicações.
	
\subsection{Agradecimentos}
Nesta secção gostaria de agradecer a todos que contribuíram para que o estágio fosse um sucesso. Em primeiro lugar, gostaria de 
agradecer ao INESC TEC pela oportunidade de realizar o estágio no laboratório SGEV, onde tive a oportunidade de aprender e crescer como 
profissional. Também gostaria de agradecer ao meu orientador na instituição, engenheiro Pedro Pascoal, pelo seu apoio e orientação ao 
longo do estágio e por me ter proporcionado os melhores meios e recursos para o desenvolvimento do trabalho, além de toda a sua paciência 
e compreensão ao longo do processo. Gostaria de agradecer ao meu orientador na FEUP/FCUP, professor Rolando Martins, pelo seu apoio e 
orientação ao longo do estágio na vertente académica e por me ter ajudado na aprovação e sugestão de melhorias no projeto. Gostaria de 
agradecer a todos os meus colegas de estágio, que me ajudaram e apoiaram ao longo do estágio, e que tornaram o ambiente de trabalho mais agradável.
Por último, gostaria de agradecer à faculdade de Engenharia da Universidade do Porto (FEUP) e à Faculdade de Ciências da Universidade do 
Porto (FCUP) pela oportunidade de desenvolver este projeto e de aprender com os melhores durante a minha licenciatura.